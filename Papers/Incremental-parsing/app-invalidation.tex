\section{Example of invalidation}

To illustrate the invalidation phase, consider the following buffer
contents:

{\small\begin{verbatim}
    ...
    34 (f 10)
    35
    36 (let ((x 1)
    37       y)
    38   (g (h x)
    39      (i y)
    40      (j x y)))
    41
    42 (f 20)
    ...
\end{verbatim}}

Each line is preceded by the absolute line number.  If the parse
result starting at line 36 is a member of the prefix or if it is the
first element of the suffix, it would be represented like this:

{\small\begin{verbatim}
    36 04 (let ((x 1) y) (g (h x) (i y) (j x y)))
       00 00 let
       00 01 ((x 1) y)
          00 00 (x 1)
             00 00 x
             00 00 1
          01 00 y
       02 02 (g (h x) (i y) (j x y))
          00 00 g
          00 00 (h x)
             00 00 h
             00 00 x
          01 00 (i y)
             00 00 i
             00 00 y
          02 00 (j x y)
             00 00 j
             00 00 x
             00 00 y
\end{verbatim}}

Since column numbers are uninteresting for our illustration, we
show only line numbers.  Furthermore, we present a list as a table for
a more compact presentation.

Suppose, for example, that the buffer contents in our example were
modified so that line 37 was altered in some way, and a line was
inserted between the lines 39 and 40.  As a result of this update, we
need to represent the following parse results:

{\small\begin{verbatim}
    ...
    34 (f 10)
    35
    36  let  (x 1)
    37
    38    g (h x)
    39      (i y)
    40
    41      (j x y)
    42
    43 (f 20)
    ...
\end{verbatim}}

In other words, we need to obtain the following representation:

{\small\begin{verbatim}
    prefix
    ...
    34 00 (f 10)
       00 00 f
       00 00 10
    residue
    36 00 let
    36 00 (x 1)
       00 00 x
       00 00 1
    38 00 g
    38 00 (h x)
       00 00 h
       00 00 x
    39 00 (i y)
       00 00 i
       00 00 y
    41 00 (j x y)
       00 00 j
       00 00 x
       00 00 y
    suffix
    43 00 (f 20)
       00 00 f
       00 00 20
    ...
\end{verbatim}}

For simplicity of presentation, we will omit the children of parse
results in the cache, and only show the top-level contents of the
prefix, the suffix, the worklist, and the residue.  When simplified,
the cache contents above is illustrated like this:

{\small\begin{verbatim}
    prefix
    ...
    34 00 (f 10)
    residue
    36 00 let
    36 00 (x 1)
    38 00 g
    38 00 (h x)
    39 00 (i y)
    41 00 (j x y)
    suffix
    43 00 (f 20)
    ...
\end{verbatim}}

The first operation to process as part of the invalidation phase is
the reported modification of line 37.  Therefore our first task is to
adjust the prefix and the suffix so that the prefix contains the last
parse result that is unaffected by the modifications.  This adjustment
results in the following situation:

{\small\begin{verbatim}
    prefix
    ...
    34 00 (f 10)
    residue
    worklist
    suffix
    36 04 (let ((x 1) y) (g (h x) (i y) (j x y)))
    06 00 (f 20)
    ...
\end{verbatim}}

We now start processing the modification of line 37.  The worklist is
empty, so the first action consists of moving the first parse result
from the suffix to the worklist, resulting in the following situation:

{\small\begin{verbatim}
    prefix
    ...
    34 00 (f 10)
    residue
    worklist
    36 04 (let ((x 1) y) (g (h x) (i y) (j x y)))
    suffix
    42 00 (f 20)
    ...
\end{verbatim}}

The first parse result of the worklist is affected by the fact that
line 37 has been modified.  We must remove that parse result from the
worklist, and push its children to the worklist.  In doing so, we make
the \texttt{start-line} of the children reflect the absolute line
number.  We obtain the following situation:

{\small\begin{verbatim}
    prefix
    ...
    34 00 (f 10)
    residue
    worklist
    36 00 let
    36 01 ((x 1) y)
    38 02 (g (h x) (i y) (j x y))
    suffix
    42 00 (f 20)
    ...
\end{verbatim}}

The first parse result of the worklist is unaffected by the fact that
line 37 has been modified, because it precedes the modified line
entirely.  We therefore move it to the residue, thereby obtaining the
following situation:

{\small\begin{verbatim}
    prefix
    ...
    34 00 (f 10)
    residue
    36 00 let
    worklist
    36 01 ((x 1) y)
    38 02 (g (h x) (i y) (j x y))
    suffix
    42 00 (f 20)
    ...
\end{verbatim}}

The first element of the worklist is affected by the modification of
line 37.  We therefore remove it from the worklist, and push its
children onto the worklist.  In doing so, we make the
\texttt{start-line} of the children reflect the absolute line number.
We obtain the following situation:

{\small\begin{verbatim}
    prefix
    ...
    34 00 (f 10)
    residue
    36 00 let
    worklist
    36 00 (x 1)
    37 00 y
    38 02 (g (h x) (i y) (j x y))
    suffix
    42 00 (f 20)
    ...
\end{verbatim}}

The first parse result of the worklist is unaffected by the fact that
line 37 has been modified, because it precedes the modified line
entirely.  We therefore move it to the residue, thereby obtaining the
following situation:

{\small\begin{verbatim}
    prefix
    ...
    34 00 (f 10)
    residue
    36 00 let
    36 00 (x 1)
    worklist
    37 00 y
    38 02 (g (h x) (i y) (j x y))
    suffix
    42 00 (f 20)
    ...
\end{verbatim}}

The first parse result of the top of the worklist is affected by the
modification.  It has no children, so we pop it off the worklist.  We
obtain the following situation:

{\small\begin{verbatim}
    prefix
    ...
    34 00 (f 10)
    residue
    36 00 let
    36 00 (x 1)
    worklist
    38 02 (g (h x) (i y) (j x y))
    suffix
    42 00 (f 20)
    ...
\end{verbatim}}

The modification of line 37 is now entirely processed.  We know this
because the first parse result on the worklist occurs beyond the
modified line in the buffer.  We therefore start processing the line
inserted between the existing lines 39 and 40.  The first item on the
worklist is affected by this insertion.  We therefore remove it from
the worklist and push its children instead.  In doing so, we make the
\texttt{start-line} of those children reflect the absolute line
number.  We obtain the following result:

{\small\begin{verbatim}
    prefix
    ...
    34 00 (f 10)
    residue
    36 00 let
    36 00 (x 1)
    worklist
    38 00 g
    38 00 (h x)
    39 00 (i y)
    40 00 (j x y)
    suffix
    42 00 (f 20)
    ...
\end{verbatim}}

The first element of the worklist is unaffected by the insertion
because it precedes the inserted line entirely.  We therefore move it
to the residue list.  We now have the following situation:

{\small\begin{verbatim}
    prefix
    ...
    34 00 (f 10)
    residue
    36 00 let
    36 00 (x 1)
    38 00 g
    worklist
    38 00 (h x)
    39 00 (i y)
    40 00 (j x y)
    suffix
    42 00 (f 20)
    ...
\end{verbatim}}

The first element of the worklist is unaffected by the insertion
because it precedes the inserted line entirely.  We therefore move it
to the residue list.  We now have the following situation:

{\small\begin{verbatim}
    prefix
    ...
    34 00 (f 10)
    residue
    36 00 let
    36 00 (x 1)
    38 00 g
    38 00 (h x)
    worklist
    39 00 (i y)
    40 00 (j x y)
    suffix
    42 00 (f 20)
    ...
\end{verbatim}}

Once again, the first element of the worklist is unaffected by the
insertion because it precedes the inserted line entirely.  We
therefore move it to the residue list.  We now have the following
situation:

{\small\begin{verbatim}
    prefix
    ...
    34 00 (f 10)
    residue
    36 00 let
    36 00 (x 1)
    38 00 g
    38 00 (h x)
    39 00 (i y)
    worklist
    40 00 (j x y)
    suffix
    42 00 (f 20)
    ...
\end{verbatim}}

The first element of the worklist is affected by the insertion, in
that it must have its line number incremented.  In fact, every element
of the worklist and also the first element of the suffix must have
their line numbers incremented.  Furthermore, this update finishes the
processing of the inserted line.  We now have the following situation:

{\small\begin{verbatim}
    prefix
    ...
    34 00 (f 10)
    residue
    36 00 let
    36 00 (x 1)
    38 00 g
    38 00 (h x)
    39 00 (i y)
    worklist
    41 00 (j x y)
    suffix
    43 00 (f 20)
    ...
\end{verbatim}}

With no more buffer modifications to process, we terminate the
procedure by moving remaining parse results from the worklist to the
residue list.  The final situation is shown here:

{\small\begin{verbatim}
    prefix
    ...
    34 00 (f 10)
    residue
    36 00 let
    36 00 (x 1)
    38 00 g
    38 00 (h x)
    39 00 (i y)
    41 00 (j x y)
    worklist
    suffix
    43 00 (f 20)
    ...
\end{verbatim}}

