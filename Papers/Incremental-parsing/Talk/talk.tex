\documentclass{beamer}
\usepackage[latin1]{inputenc}
\beamertemplateshadingbackground{red!10}{blue!10}
%\usepackage{fancybox}
\usepackage{epsfig}
\usepackage{verbatim}
\usepackage{url}
\usepackage{graphics}
%\usepackage{xcolor}
\usepackage{fancybox}
\usepackage{moreverb}
%\usepackage[all]{xy}
\usepackage{listings}
\usepackage{filecontents}
\usepackage{graphicx}
\usepackage{alltt}

\lstset{
  language=Lisp,
  basicstyle=\scriptsize\ttfamily,
  keywordstyle={},
  commentstyle={},
  stringstyle={}}

\def\inputfig#1{\input #1}
\def\inputeps#1{\includegraphics{#1}}
\def\inputtex#1{\input #1}

\inputtex{logos.tex}

%\definecolor{ORANGE}{named}{Orange}

\definecolor{GREEN}{rgb}{0,0.8,0}
\definecolor{YELLOW}{rgb}{1,1,0}
\definecolor{ORANGE}{rgb}{1,0.647,0}
\definecolor{PURPLE}{rgb}{0.627,0.126,0.941}
\definecolor{LIGHTPURPLE}{rgb}{0.627,0.200,0.941}
\definecolor{PURPLE}{named}{purple}
\definecolor{PINK}{rgb}{1,0.412,0.706}
\definecolor{WHEAT}{rgb}{1,0.8,0.6}
\definecolor{BLUE}{rgb}{0,0,1}
\definecolor{GRAY}{named}{gray}
\definecolor{CYAN}{named}{cyan}

\newcommand{\orchid}[1]{\textcolor{Orchid}{#1}}
\newcommand{\defun}[1]{\orchid{#1}}

\newcommand{\BROWN}[1]{\textcolor{BROWN}{#1}}
\newcommand{\RED}[1]{\textcolor{red}{#1}}
\newcommand{\YELLOW}[1]{\textcolor{YELLOW}{#1}}
\newcommand{\PINK}[1]{\textcolor{PINK}{#1}}
\newcommand{\WHEAT}[1]{\textcolor{wheat}{#1}}
\newcommand{\GREEN}[1]{\textcolor{GREEN}{#1}}
\newcommand{\PURPLE}[1]{\textcolor{PURPLE}{#1}}
\newcommand{\LIGHTPURPLE}[1]{\textcolor{LIGHTPURPLE}{#1}}
\newcommand{\BLACK}[1]{\textcolor{black}{#1}}
\newcommand{\WHITE}[1]{\textcolor{WHITE}{#1}}
\newcommand{\MAGENTA}[1]{\textcolor{MAGENTA}{#1}}
\newcommand{\ORANGE}[1]{\textcolor{ORANGE}{#1}}
\newcommand{\BLUE}[1]{\textcolor{BLUE}{#1}}
\newcommand{\GRAY}[1]{\textcolor{gray}{#1}}
\newcommand{\CYAN}[1]{\textcolor{cyan }{#1}}

\newcommand{\reference}[2]{\textcolor{PINK}{[#1~#2]}}
%\newcommand{\vect}[1]{\stackrel{\rightarrow}{#1}}

% Use some nice templates
\beamertemplatetransparentcovereddynamic

\newcommand{\A}{{\mathbb A}}
\newcommand{\degr}{\mathrm{deg}}

\title{Incremental Parsing of Common Lisp Code}

\author{Ir�ne Durand \& Robert Strandh}
\institute{
LaBRI, University of Bordeaux
}
\date{April, 2018}

%\inputtex{macros.tex}


\begin{document}
\frame{
\resizebox{3cm}{!}{\includegraphics{Logobx.pdf}}
\hfill
\resizebox{1.5cm}{!}{\includegraphics{labri-logo.pdf}}
\titlepage
\vfill
\small{European Lisp Symposium, Marbella, Spain \hfill ELS2018}
}

\setbeamertemplate{footline}{
\vspace{-1em}
\hspace*{1ex}{~} \GRAY{\insertframenumber/\inserttotalframenumber}
}

\frame{
\frametitle{Context}

Emacs is likely the most common editor for Common Lisp code.

\begin{itemize}
\item The current package is not taken into account.
\item The indent function can not distinguish between forms and
  bindings.
\item No distinction between different roles of symbols.
\item Incorrect indentation is not indicated.
\end{itemize}
}

\begin{frame}[fragile]
\frametitle{Taking packages into account}
Emacs does not take packages into account for syntax highlighting.
\vskip 0.5cm
This code is highlighted correctly:

\begin{alltt}
(\LIGHTPURPLE{defpackage} :p (:use :common-lisp))

(\LIGHTPURPLE{in-package} :p)

(\LIGHTPURPLE{defun} f (x) x)
\end{alltt}
\end{frame}

\begin{frame}[fragile]
\frametitle{Taking packages into account}
Emacs does not take packages into account for syntax highlighting.
\vskip 0.5cm
This code is \RED{not} highlighted correctly:
\begin{alltt}
(\LIGHTPURPLE{defpackage} :p (:use))

(\LIGHTPURPLE{in-package} :p)

(\LIGHTPURPLE{defun} f (x) x)
\end{alltt}
\end{frame}

\begin{frame}[fragile]
\frametitle{Distinguishing between forms and bindings}
Emacs does not distinguish between forms and bindings.
\vskip 0.5cm
This binding is indented in one way:
\begin{alltt}
(\LIGHTPURPLE{let} ((temp
        (find key *entries* :test #'eq :key #'car)))
  ...)
\end{alltt}
\end{frame}

\begin{frame}[fragile]
\frametitle{Distinguishing between forms and bindings}
Emacs does not distinguish between forms and bindings.
\vskip 0.5cm
This binding is indented in a different way:
\begin{alltt}
(\LIGHTPURPLE{let} ((\LIGHTPURPLE{prog1}
          (find key *entries* :test #'eq :key #'car)))
  ...)
\end{alltt}

And the role of \texttt{prog1} is not taken into account.
\end{frame}

\begin{frame}[fragile]
\frametitle{Indicating incorrect indentation}
Emacs does not indicate bad indentation.
\vskip 0.5cm
This form contains an incorrect indentation:
\begin{alltt}
(\LIGHTPURPLE{let*} ((x (expt *result* 3))
  (\LIGHTPURPLE{declare} (float x)))
  (+ x 1.0))
\end{alltt}
\end{frame}

\begin{frame}
\frametitle{Objectives}

An excellent editor for \commonlisp{} code:
\vskip 0.5cm
\begin{itemize}
\item Take current package into account.
\item Distinguish froms from other entities.
\item Show incorrect indentation.
\item Take roles of symbols into account.
\item Provide refactoring functionality.
\end{itemize}
\end{frame}

\begin{frame}
\frametitle{First step towards objectives}
Create an incremental parser for \commonlisp{} code that yields a
considerably more accurate result than existing parsers.
\end{frame}

\begin{frame}
\frametitle{Recapitulation: Editor buffer protocol}
Presented at ELS 2016.
\vskip 0.5cm
Two sub-protocols:
\vskip 0.5cm
\begin{itemize}
\item Edit protocol.  Access, insert, or delete an item.  Can be
  invoked a large number of times for each keystroke.
\item Update protocol.  Determine changes since last update.
  Typically invoked one time for each keystroke.
\end{itemize}
\vskip 0.5cm
For the current work, we are only interested in the update protocol.
\end{frame}

\begin{frame}
\frametitle{Our technique (modified \texttt{read})}

We use a modified version of the standard Common Lisp function
\texttt{read}:
\begin{itemize}
\item It returns \emph{parse results} that are expressions wrapped in
  standard instances containing source location.
\item It also returns non-expressions like comments and other material
  for which reader macros return no values.
\end{itemize}
\end{frame}

\begin{frame}
\frametitle{Our technique (Gray stream for text buffer)}

The modified \texttt{read} function uses a Gray stream that accesses
the contents of the text buffer.
\end{frame}

\begin{frame}
\frametitle{Our technique (cache of parse results)}

We maintain a \emph{cache} that maps source locations to parse results.
\vskip 0.5cm
When there is an attempt to call \texttt{read} at a position that has
an entry in the cache, the cached parse result is returned and the stream is
advanced past the parse result.
\end{frame}

\frame{
\frametitle{Performance}
Tests run on a $4$-core Intel Core processor clocked at $3.3$GHz,
running SBCL version 1.3.11.
}


\begin{frame}
\frametitle{Performance \small{Inserting and deleting a constituent character}}

\begin{center}
\small{
\begin{tabular}{|r|r|r|}
\hline
nb forms & form size & time\\
\hline
120 & 10 & 0.14ms\\
80 & 15  & 0.14ms\\
60 & 20  & 0.14ms\\
24 & 100 & 0.23ms\\
36 & 100 & 0.32ms\\
\hline
\end{tabular}
}
\end{center}
\end{frame}

\begin{frame}
\frametitle{Performance \small{Inserting and deleting a constituent character}}

\begin{center}
\includegraphics[width=7cm]{Curves/insert-delete-x.eps}
\end{center}
\end{frame}

\begin{frame}
\frametitle{Performance \small{Inserting and deleting a left parenthesis}}

\begin{center}
{\small
    \begin{tabular}{|r|r|r|}
      \hline
      nb forms & form size & time\\
      \hline
      120 & 10 & 1.3ms\\
      80 & 15  & 1.0ms\\
      60 & 20  & 0.5ms\\
      40 & 30  & 0.7ms\\
      30 & 40  & 0.6ms\\
      24 & 50  & 0.5ms\\
      12 & 100 & 0.5ms\\
      \hline
    \end{tabular}
}
\end{center}
\end{frame}
 
\begin{frame}
\frametitle{Performance \small{Inserting and deleting a left parenthesis}}
\begin{center}
\includegraphics[width=7cm]{Curves/insert-delete-parenthesis.eps}
\end{center}
\end{frame}

\begin{frame}
\frametitle{Performance \small{Inserting and deleting a double quote}}

\begin{center}
{\small
    \begin{tabular}{|r|r|r|r|}
      \hline
      nb forms & form size & characters per line & time\\
      \hline
      120 & 10 & 1 & 18ms\\
      80 & 15  & 1 & 15ms\\
      60 & 20  & 1 & 17ms\\
      24 & 100 & 1 & 33ms\\
      36 & 100 & 1 & 50ms\\
      120 & 10 & 30 & 70ms\\
      \hline
    \end{tabular}
}
\end{center}
\end{frame}

\begin{frame}
\begin{center}
\includegraphics[width=7cm]{Curves/insert-delete-double-quote.eps}
\end{center}
\end{frame}

\frame{
\frametitle{Future work}

\begin{itemize}
\item Use parse result to compute indentation.
\item Implement incremental version of first-class global
  environments.
\item Use new environment implementation to compile top-level forms at
  typing speed.
\item Display information from compilation.
\item Implement refactoring tools based on compilation.
\end{itemize}
}

\frame{
  \frametitle{Acknowledgments}

We would like to thank Philipp Marek and Cyrus Harmon for providing
valuable feedback on early versions of this paper.
}

\frame{
\frametitle{Thank you}

Questions?
}

\end{document}
