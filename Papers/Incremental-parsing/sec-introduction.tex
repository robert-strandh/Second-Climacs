\section{Introduction}

Whether autonomous or part of an integrated development environment,
an editor that caters to \commonlisp{} programmers must analyze the
buffer contents in order to help the programmer understand how this
contents would be analyzed when submitted to the \commonlisp{}
compiler.  Furthermore, the editor analysis must be \emph{fast} so
that it is up to date shortly after each keystroke by the programmer.

In order to obtain such speed for the analysis, it must be
\emph{incremental}.  A complete analysis of the entire buffer for each
keystroke is generally not feasible, especially for buffers with a
significant amount of code.

Furthermore, the analysis is necessarily \emph{approximate}.  The
reader macro \#. (hash dot) and the macro \texttt{eval-when} allow for
arbitrary computations at read time and at compile time, and these
computations may influence the environment in arbitrary ways that may
invalidate subsequent, or even preceding analyses, making an analysis
that is both precise and incremental impossible in general.

The question, then, is how approximate the analysis has to be, and how
much of it we can allow ourselves to recompute, given the performance
of modern hardware and modern \commonlisp{} implementations.

In this paper, we describe an analysis technique that represents an
improvement compared with the ones used by the most widespread editors
for \commonlisp{} code used today.  The technique is more precise than
existing ones, because it uses the \commonlisp{} \texttt{read}
function, which is a better approximation than the regular-expression
techniques most frequently used.  We show that our analysis is
sufficiently fast because it is done incrementally, and it requires
very little incremental work for most simple editing operations.

