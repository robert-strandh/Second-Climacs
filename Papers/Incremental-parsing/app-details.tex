\section{Cache representation}
\label{app-cache-representation}

The sequence of top-level parse results is split into a \emph{prefix}
and a \emph{suffix}, typically reflecting the current position in the
buffer being edited by the end user.  The suffix contains parse
results in the order they appear in the buffer, whereas the prefix
contains parse results in reverse order, making it easy to move parse
results between the prefix and the suffix.

Depending on the location of the parse result in the cache data
structure, its position may be \emph{absolute} or \emph{relative}.
The prefix contains parse results that precede updates to the buffer.
For that reason, these parse results have absolute positions.  Parse
results in the suffix, on the other hand, follow updates to the
buffer.  In particular, if a line is inserted or deleted, the parse
results in the suffix will have their positions changed.  For that
reason, only the first parse result of the suffix has an absolute
position.  Each of the others has a position relative to its
predecessor.  When a line is inserted or deleted, only the first parse
result of the suffix has to have its position updated.  When a parse
result is moved from the prefix to the suffix, or from the suffix to
the prefix, the positions concerned are updated to maintain this
invariant.

To avoid having to traverse all the descendants of a parse result when
its position changes, we make the position of the first child of some
parse result $P$ relative to that of $P$, and the children, other than
the first, of some parse result $P$, have positions relative to the
previous child in the list.
