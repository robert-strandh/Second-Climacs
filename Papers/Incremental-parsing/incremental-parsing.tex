\documentclass{sig-alternate-05-2015}
\usepackage[utf8]{inputenc}
\usepackage{color}

\def\inputfig#1{\input #1}
\def\inputtex#1{\input #1}
\def\inputal#1{\input #1}
\def\inputcode#1{\input #1}

\inputtex{logos.tex}
\inputtex{refmacros.tex}
\inputtex{other-macros.tex}

\begin{document}
\setcopyright{rightsretained}
\title{Incremental Parsing of Common Lisp Code}
\numberofauthors{2}
\author{\alignauthor
Irène Durand\\
Robert Strandh\\
\affaddr{University of Bordeaux}\\
\affaddr{351, Cours de la Libération}\\
\affaddr{Talence, France}\\
\email{irene.durand@u-bordeaux.fr}
\email{robert.strandh@u-bordeaux.fr}}

\toappear{Permission to make digital or hard copies of all or part of
  this work for personal or classroom use is granted without fee
  provided that copies are not made or distributed for profit or
  commercial advantage and that copies bear this notice and the full
  citation on the first page. Copyrights for components of this work
  owned by others than the author(s) must be honored. Abstracting with
  credit is permitted. To copy otherwise, or republish, to post on
  servers or to redistribute to lists, requires prior specific
  permission and/or a fee. Request permissions from
  Permissions@acm.org.

  ELS '17, April 3 -- 6 2017, Brussels, Belgium
  Copyright is held by the owner/author(s). %Publication rights licensed to ACM.
%  ACM 978-1-4503-2931-6/14/08\$15.00.
%  http://dx.doi.org/10.1145/2635648.2635654
}

\maketitle

\begin{abstract}
In a text editor for writing \commonlisp{} \cite{ansi:common:lisp}
source code, it is desirable to have an accurate analysis of the
buffer contents, so that the role of the elements of the code can be
indicated to the programmer.  Furthermore, the buffer contents should
preferably be analyzed after each keystroke so that the programmer has
up-to-date information resulting from the analysis.

We describe an incremental parser that can be used as a key component
of such an analyzer.  The parser, itself written in \commonlisp{},
uses a special-purpose implementation of the \commonlisp{}
\texttt{read} function in combination with a \emph{cache} that stores
existing results of calling the reader.

Since the parser uses the standard \commonlisp{} reader, the resulting
analysis is very accurate.  Furthermore, the cache makes the parser
very fast in most common cases; re-parsing a buffer in which a single
character has been altered takes only a few milliseconds.
\end{abstract}

\begin{CCSXML}
  <ccs2012>
  <concept>
  <concept_id>10011007.10011006.10011066.10011069</concept_id>
  <concept_desc>Software and its engineering~Integrated and visual development environments</concept_desc>
  <concept_significance>500</concept_significance>
  </concept>
  </ccs2012>
\end{CCSXML}

\ccsdesc[500]{Software and its engineering~Integrated and visual development environments}

\printccsdesc

\keywords{\commonlisp{}, text editing}

\inputtex{sec-introduction.tex}
\inputtex{sec-previous.tex}
\inputtex{sec-our-method.tex}
\inputtex{sec-performance.tex}
\inputtex{sec-conclusions.tex}
\inputtex{sec-acknowledgements.tex}

\bibliographystyle{abbrv}
\bibliography{incremental-parsing}
\inputtex{app-details.tex}

\end{document}
