\chapter{Representation of the editor buffer}
\label{chap-internals-buffer}

A buffer can have any representation that respects the \emph{buffer
  protocol}.

\section{Buffer protocols}

The buffer participates in two different buffer protocols:

\begin{enumerate}
\item The \emph{edit protocol}, used by client editing and
  cursor-motion operations.
\item The \emph{query protocol}, used by redisplay operations to
  determine what items are contained in the buffer. 
\end{enumerate}

The buffer protocols exposes two levels of abstraction to client code: 

\begin{enumerate}
\item The \emph{buffer} level represents the \emph{sequence of lines}
  independently of how the individual lines are represented.  
\item The \emph{line} level represents individual lines. 
\end{enumerate}

The buffer protocols do not pretend to manage any equivalence between
line breaks and some sequence of characters.  It is up to client code
to model such an equivalence if desired.  As a consequence, the buffer
protocols do not allow for a cursor at the beginning of a line to
more backward or a cursor at the end of a line to move forward.  An
attempt at doing so will result in an error being signaled.  If client
code wants to impose a model where the line break corresponds to (say)
the \emph{newline} character, then it must explicitly detach and
reattach the cursor to a different line in these cases.  It can manage
that in two different ways: either by explicitly testing for
\texttt{beginning-of-line} or \texttt{end-of-line} before calling the
equivalent buffer function, or by handling the error that results from
the attempt. 

The buffer also does not interpret the meaning of any of items
contained in it.  For instance, whether an item is to be considered
part of a \emph{word} or not, is not decided at the buffer level, but
at the level of the syntax.  As a consequence, the buffer protocol
does not offer any functions that require such interpretation, such as
\texttt{forward-word}, \texttt{end-of-paragraph}, etc. 

\Defun {beginning-of-buffer-p} {cursor}

Return \textit{true} if and only if \textit{cursor} is located at the
beginning of a buffer.

\Defun {end-of-buffer-p} {cursor}

Return \textit{true} if and only if \textit{cursor} is located at the
end of a buffer.

\Defun {beginning-of-line-p} {cursor}

Return \textit{true} if and only if \textit{cursor} is located at the
beginning of a line.

\Defun {end-of-line-p} {cursor}

Return \textit{true} if and only if \textit{cursor} is located at the
end of a line.

\Defun {beginning-of-buffer} {cursor}

Position \textit{cursor} at the very beginning of the buffer.

\Defun {end-of-buffer} {cursor}

Position \textit{cursor} at the very end of the buffer.

\Defun {forward-item} {cursor}

Move \textit{cursor} forward one position.  If \emph{cursor} is at the
end of the line, the error condition \texttt{end-of-line} will be
signaled.

\Defun {backward-item} {cursor}

Move \textit{cursor} backward one position.  If \emph{cursor} is at
the beginning of the line, the error condition
\texttt{beginning-of-line} will be signaled.

\Defun {insert-item} {cursor item}

Insert an item at the location of \textit{cursor}.  After this
operation, any left-sticky cursor located at the same position as
\textit{cursor} will be located before \textit{item}, and any
right-sticky cursor located at the same position as \textit{cursor}
will be located after \textit{item}.

\Defun {delete-item} {cursor}

Delete the item immediately to the right of \emph{cursor}.  If
\emph{cursor} is at the end of the line, the error condition
\texttt{end-of-line} will be signaled.

\Defun {erase-item} {cursor}

Delete the item immediately to the left of \emph{cursor}.  If
\emph{cursor} is at the beginning of the line, the error condition
\texttt{beginning-of-line} will be signaled.

\Defun {item-after-cursor} {cursor}

Return the item located immediately to the right of \textit{cursor}.
If \emph{cursor} is at the end of the line, the error condition
\texttt{end-of-line} will be signaled.

\Defun {item-before-cursor} {cursor}

Return the item located immediately to the left of \textit{cursor}.
If \emph{cursor} is at the beginning of the line, the error condition
\texttt{beginning-of-line} will be signaled.

\Defun {split-line} {cursor}

Split the line in which \textit{cursor} is located into two lines, the
first cone containing the text preceding \textit{cursor} and the
second one containing the text following \textit{cursor}.  After this
operation, any left-sticky cursor located at the same position as
\textit{cursor} will be located at the end of the first line, and any
right-sticky cursor located at the same position as \textit{cursor}
will be located at the beginning of the second line.

\Defun {join-line} {cursor}

Join the line in which \textit{cursor} is located and the following
line.  If \textit{cursor} is located at the last line of the buffer,
the error condition \texttt{end-of-buffer} will be signaled.

\Defun {detach-cursor} {cursor}

The class of \textit{cursor} is changed to \texttt{detached-cursor}
and it is removed from the line it was initially located in. 

If \textit{cursor} is already detached, this operation has no effect.

\Defun {attach-cursor} {cursor line \optional (position 0)}

Attach \textit{cursor} to \textit{line} at \textit{position}.  If
\textit{position} is supplied and it is greater than the number of
items in \textit{line}, the error condition \texttt{end-of-line} is
signaled.  If \textit{cursor} is already attached to a line, the error
condition \texttt{cursor-attached} will be signaled.

\Defun {cursor-position} {cursor}

Return the position of \textit{cursor} as two values: the line number
and the item number within the line. 

\Defun {line-count} {cursor}

Return the number of lines in the buffer in which \textit{cursor} is
located.

\Defun {item-count} {cursor}

Return the number of items in the line in which \textit{cursor} is
located.

\section{Standard buffer representation}

The standard buffer representation organizes the lines in a
\emph{splay tree} \cite{Sleator:1985:SBS:3828.3835}.  This
organization has several advantages:

\begin{itemize}
\item A line that is modified moves to the root of the tree, and
  recently used lines stay close to the root, making some editing
  operations more efficient.
\item It is computationally cheap to know the line number of the
  current line at all times. 
\end{itemize}

\refFig{fig-buffer} illustrates this representation.

\begin{figure}
\begin{center}
\inputfig{fig-buffer.pdf_t}
\end{center}
\caption{\label{fig-buffer}
Representation of the buffer as a splay tree.}
\end{figure}


The standard buffer representation uses two different representations
of a \emph{line}.  Whenever a line is modified, it must be
\emph{open}.  In the standard buffer representation, an open line is
represented as a \emph{splay tree} \cite{Sleator:1985:SBS:3828.3835}
of \emph{items}.  Items are typically characters, but any object is
allowed.

