\chapter{Representation of the editor buffer}
\label{chap-internals-buffer}

The standard buffer representation organizes the lines in a
\emph{splay tree} \cite{Sleator:1985:SBS:3828.3835}.  This
organization has several advantages:

\begin{itemize}
\item A line that is modified moves to the root of the tree, and
  recently used lines stay close to the root, making some editing
  operations more efficient.
\item It is computationally cheap to know the line number of the
  current line at all times. 
\end{itemize}

The standard buffer representation uses two different representations
of a \emph{line}.  Whenever a line is modified, it must be
\emph{open}.  An open line is represented as a \emph{gap buffer}.  A
closed line is currently represented as a simple vector, but a more
compact representation could be chosen based on the contents of the
line.

A line is closed whenever its contents is asked for, using the
function \code{items}.  Then, if the entire contents is asked for, the
simple vector is simply returned.  Otherwise, the result of calling
\code{subseq} on the simple vector is returned. 

The fact that a line is closed whenever its contents is asked for may
seem wasteful, because for a simple editing operation such as
inserting or deleting an item, the line will be opened and closed for
every key stroke.  However, this system was not designed to optimize
that particular use case.  When a line is opened and closed at the
frequency of a key stroke, it will typically take very little CPU time
(as a percentage of full CPU use) because of the relative slowness of
the human operator.  Instead it was optimized for the use case when
several simple editing operations result from a small number of
keystrokes.  A typical example would be loading a file into a buffer
or executing a keyboard macro that inserts several items.  For this
use case, the affected lines typically remain open for the duration of
the interaction, and they are closed only as a result of the result
being displayed. 

A \emph{cursor} can be either \emph{detached} or \emph{attached} as
reflected by its \emph{class}.  When a cursor needs to move from one
line to another line, it is first detached from its current line and
then attached to its target line.  An attached cursor contains a slot
that refers to the line to which it is attached, and a slot that
contains its \emph{logical position} within the line.  A cursor can
also be either open or closed, depending on whether it is in an open
line or a closed line.  Currently, there is no difference between the
representation of an open cursor and of a closed cursor.  In
applications where a line can have many (i.e., hundreds) of cursors,
it might be good to store the \emph{physical position} in the gap
buffer.  This way, cursor positions will only have to be updated when
the gap moves, as opposed to after every editing operation, at least
for insert operations.  
