\chapter{General control structure}
\label{chap-internals-control-structure}

The general control structure was designed with the following goals: 

\begin{itemize}
\item Most editing operations should be very fast, even when they
  involve fairly large chunks of buffer contents.  Here, \emph{fast}
  means that the response time for interactive editing should be
  short. 
\item From a software-engineering point of view, the buffer editing
  operations should not be aware of the presence of any \emph{views}. 
\end{itemize}

Notice that it was \emph{not} a goal that editing operations use as
little computational power as possible.  

Input events can be divided into two categories:

\begin{itemize}
\item Input events that result in some modification to some buffer
  contents.  Inserting and deleting items are in this category.
  Modifications can be the result of indirect events such as executing
  a keyboard macro that inserts or deletes items in one or more
  buffers. 
\item Input events that have no effect on any buffer contents.  Moving
  a cursor, changing the size of a window, or scrolling a view are
  typical events in this category.  These events influence only the
  \emph{view} into a buffer.
\end{itemize}

When an event in the first category occurs, the following chain of
events is triggered:

\begin{enumerate}
\item The event itself triggers the execution of some \emph{command}
  that causes one or more items to be inserted and/or deleted from one
  or more buffers.  Whether this happens as a direct result or as an
  indirect result of the event makes no difference.  The buffers
  involved are modified, but no other action is taken at this time.
  Lines that are modified or inserted are marked with the
  \emph{current time stamp} and the current time stamp is
  incremented, possibly more than once. 
\item At the end of the execution of the command, the \emph{syntax
  update} is executed for all buffers, allowing the contents to be
  incrementally parsed according to the syntax associated with the
  buffer.  Finally, visible views are repainted using whatever
  combination they want of the buffer contents and the result of the
  syntax update.  The syntax update uses the time stamps of lines in
  the buffer and of the previous syntax update to compute an
  up-to-date representation of the buffer.  This computation is done
  incrementally as much as possible.
\item Each view on display recomputes the data presented to the user
  and redraws the associated window.  Again, time stamps are used to
  make this computation as incremental as possible.
\end{enumerate}
