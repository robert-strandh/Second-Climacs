\chapter{Introduction}
\pagenumbering{arabic}

\sysname{} is an \emacs{}-like editor written entirely in
\commonlisp{}.  It is called \sysname{} because it is a complete
rewrite of the \climacs{} text editor.

\climacs{} gave us some significant experience with writing a text
editor, and we think we can improve on a number of aspects of it.  As
a result, there are some major differences between \climacs{} and
\sysname{}:

\begin{itemize}
\item We implemented a better buffer representation, and extracted it
  from the editor code into a separate library named \cluffer{}.  The
  new buffer representation will have better performance, especially
  on large buffers, and it will make it easier to write sophisticated
  parsers for buffer contents.
\item The incremental parser for \commonlisp{} syntax of \climacs{} is
  very hard to maintain, and while it is better than that of \emacs{},
  it is still not good enough.  \sysname{} uses a modified version of
  the \commonlisp{} reader in order to parse buffer contents, making
  it much closer to the way the contents is read by the \commonlisp{}
  compiler.
\item \climacs{} depends on \mcclim{} for its graphic user interface.
  \sysname{} is independent of any particular library for making
  graphic user interfaces, allowing it to be configured with different
  such libraries.
\end{itemize}
