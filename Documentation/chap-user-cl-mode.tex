\chapter{\cl{} mode}

The contents of this chapter was largely written long before the \cl{}
mode was implemented or even fully designed.  We still use the present
tense to describe functionality, though, as if it existed. 

\section{Syntax-related information}

Syntax-related information is presented as slight alterations of how
the code elements in the buffer is displayed to the user.  The
following methods are used (roughly in decreasing order of frequency):

\begin{itemize}
\item Changing the background color.
\item Changing the foreground color.
\item Changing the font face.
\item Using a different glyph.
\end{itemize}

In addition, a code element that is marked in this way also has
associated textual information that can be read in the minibuffer when
the cursor is positioned on the text, or as a tooltip when the pointer
is positioned above the text.  In the remainder of this chapter, we
give the English version of the textual information.
Internationalization may change the language according to user
preference.

Furthermore, in some cases a code element that is marked this way has
a \emph{context menu} associated with it.

\subsection{Information about symbols}

\subsubsection{Invalid symbol syntax}

A token which does not have any interpretation as a number is
considered a potential symbol.  If so, there are a few cases where
token is nevertheless invalid as the name of a symbol.

The token might have an invalid constellation of package markers:

\begin{itemize}
\item Too many package markers.
\item Two package markers that are separated by some other character. 
\item A single package marker at the end of the token.
\end{itemize}

In this case, the background color of the package markers is vivid
red, and the background of the symbol itself is a lighter red. 

The associated textual information says ``Illegal constellation of
package markers''. 

\refFig{fig-invalid-package-markers} illustrates how this type of
information is displayed.

\begin{figure}
\begin{center}
\inputfig{fig-invalid-package-markers.pdf_t}
\end{center}
\caption{\label{fig-invalid-package-markers}
Display of potential symbol with illegal package markers.}
\end{figure}

No context menu is suggested. 

\subsubsection{Non-existing package}

A symbol with one or two package markers but where the package
indicated in the prefix does not exists is marked with a pink
background, and the package name is marked with a vivid red
background. 

The associated textual information says ``Non-existing package''.

\refFig{fig-non-existing-package} illustrates how this type of
information is displayed.

\begin{figure}
\begin{center}
\inputfig{fig-non-existing-package.pdf_t}
\end{center}
\caption{\label{fig-non-existing-package}
Display of potential symbol with illegal package markers.}
\end{figure}

The context menu has a single option: ``Create the package''.

\subsubsection{Non-existing symbol}

A symbol token with a single package marker that refers to a
non-existing symbol is marked with a vivid red background.  The
associated textual information says ``Non-existing symbol''.  

The context menu gives the following options:

\begin{itemize}
\item Create and export the symbol in the specified package.
\item Import the symbol from a different package, and export it from
  the specified package.  The user will be prompted for the package to
  import from. 
\end{itemize}

\subsubsection{Unexported symbol}

A symbol token with a single package marker that refers to an
existing, but unexported symbol is marked with a vivid
red background.  The associated textual information says ``The symbol
is not exported''.

The context menu gives the following options:

\begin{itemize}
\item Export the symbol from the specified package.
\end{itemize}

\section{Indentation}

\section{Comments}

%%  LocalWords:  tooltip minibuffer unexported
