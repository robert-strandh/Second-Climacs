\chapter{General structure}
\label{chap-general-structure}

At any point in time \sysname{} contains a certain number of
\emph{buffers}.  A buffer can only be modified as a result of the
execution of a \emph{command}.  Such a command can be executed as a
result of a \emph{key sequence}, of typing the \emph{name} of the
command into a command-line command processor, or of clicking on a
button or a menu entry that was designed to execute a command.
Generally speaking, most events, such as scrolling or resizing a
window, do not result in a command being executed. 

Before a buffer can be viewed, its \emph{syntax} must be analyzed by a
\emph{syntax analyzer}.  Normally, there is a single syntax analyzer
for a buffer, and it is chosen based on the contents of the buffer,
which is typically determined by the extension of the file from which
the buffer was created.  Thus, for instance, a buffer containing \cl{}
code is typically analyzed using the analyzer for \cl{} code.
However, it is possible for a single buffer to have several different
simultaneous syntax analyzers.  The syntax analyzers of a buffer
generally analyze the syntax \emph{incrementally} whenever possible.
This incremental analysis is triggered by the command loop after the
execution of a command.  Every syntax analyzer is triggered after each
iteration of the command loop, though typically most buffers have not
been modified so the incremental analysis then terminates
immediately.  Though in most cases a syntax analyzer refers to a
single buffer, it is possible for a syntax analyzer to refer to
several buffer, or even to other syntax analyzers.  

All user interaction, both input and output, is mediated through a
\emph{view}.  The view typically contains a single syntax analyzer
which in turn contains a single buffer.  More complicated views can
contain several syntax analyzers.  Each view typically contains a
\emph{cursor} into the buffer of its syntax analyzer.  Each view also
contains a \emph{command processor} to which key strokes are directed
when the view is the \emph{current view}.  These key strokes may
modify the buffer(s) associated with the view, or they may influence
the view itself in some way (moving the cursor for instance).

A view may or may not be \emph{visible}.  Views that are not visible
will still have they associated syntax analyzers updated after each
iteration around the command loop, but the processing stops there.
Views that are visible also have a \emph{show} associated with them.
They show is charged with transforming the result of the syntax
analysis to elements of the graphic user interface.  In a way
analogous to that of a syntax analyzer, a show is updated
incrementally from the result of the syntax analyses of the associated
view.  However, whereas the syntax analyzer is updated at each
iteration of the command loop, the show is updated at each iteration
of the \emph{event loop}.  The reason for this more frequent update is
that the show might have to change as a result of events such as
resizing a window or scrolling.  Because the data structures of a show
can take up considerable space, they are discarded when a view is no
longer visible, and they are recomputed when the view again becomes
visible. 

%%  LocalWords:  resizing
